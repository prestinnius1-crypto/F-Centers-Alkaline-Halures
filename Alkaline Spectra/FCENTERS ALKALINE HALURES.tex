\documentclass[10pt,a4paper,twocolumn]{article}

% =========================================
% Idioma y codificación
% =========================================
\usepackage[utf8]{inputenc}
\usepackage[spanish]{babel}   % español
\usepackage[T1]{fontenc}

% =========================================
% Paquetes matemáticos
% =========================================
\usepackage{amsmath, amssymb, amsfonts} % entornos matemáticos
\usepackage{physics}       % notación de física
\usepackage{bm}            % vectores en negrita
\usepackage{siunitx}       % unidades SI

% =========================================
% Paquetes gráficos y tablas
% =========================================
\usepackage{graphicx}      % imágenes
\usepackage{caption}
\usepackage{subcaption}    % subfiguras
\usepackage{float}         % [H] para fijar posición
\usepackage{booktabs}      % tablas bonitas
\usepackage{multirow}      % celdas unidas
\usepackage{xcolor}        % colores

% =========================================
% Estilo de página y columnas
% =========================================
\usepackage[
    a4paper,
    top=2.5cm,
    bottom=3cm,
    left=3.75cm,
    right=3.75cm
]{geometry}                % márgenes ajustados

\setlength{\columnsep}{0.7cm} % separación entre columnas

% =========================================
% Bibliografía
% =========================================
\usepackage[numbers,sort&compress]{natbib} % estilo numérico, comprimido
\bibliographystyle{unsrtnat}               % referencias orden de aparición

% =========================================
% Numeración de secciones en romano
% =========================================
\renewcommand{\thesection}{\Roman{section}}
\renewcommand{\thesubsection}{\thesection.\arabic{subsection}}

% =========================================
% Documento
% =========================================
\begin{document}

% Título y autores
\title{\bfseries Análisis espectral de cianinas: aproximación mediante pozos de potencial}
\author{Renato Cardelli, Elisa Puebla, Martín Ariel Zárate Lipovetzky \\ 
\small Departamento de Física, Universidad Nacional de La Plata (1900)}
\date{} % sin fecha
\maketitle

% Resumen
\begin{abstract}
En este trabajo se estudian las propiedades espectrales de cianinas mediante una aproximación de pozo de potencial infinito y comparación con datos experimentales. 
\end{abstract}

% Introducción
\section{Introducción}
Un Quantum Dot (QD) es un sólido semiconductor a escala nanométrica cuyas propiedades electrónicas dependen de su tamaño y forma. 
En este trabajo consideramos el caso de pozos de potencial esféricos. 

% Ejemplo ecuación
\begin{equation}
    E_g(R) = E_b + \frac{\hbar^2 \pi^2}{2 R^2} \left( \frac{1}{m_e} + \frac{1}{m_h} \right) - \frac{1.8 \, e^2}{\kappa R}
\end{equation}


% Reultados
\section{Resultados}
Los resultados muestran un buen acuerdo entre el modelo de pozo infinito y los radios obtenidos mediante SAXS y espectroscopía.

% Ejemplo tabla
\begin{table}[H]
\centering
\begin{tabular}{ccc}
\toprule
Muestra & $R_{\text{SAXS}}$ [nm] & $R_{\text{Abs}}$ [nm] \\
\midrule
1 & 3.6 & 2.5 \\
2 & 4.5 & 2.8 \\
3 & 3.0 & 2.3 \\
\bottomrule
\end{tabular}
\caption{Radios obtenidos por ambas técnicas.}
\label{tab:radios}
\end{table}

% Conclusión
\section{Conclusión}
El análisis muestra que las propiedades ópticas de los QD dependen fuertemente de su tamaño, verificando la teoría de confinamiento cuántico.

% Bibliografía
\bibliography{bibliografia} % archivo bibliografia.bib

\end{document}
